\documentclass[10pt]{article}
\usepackage{graphicx}  % For including the image
\usepackage{geometry}  % For setting margins
\usepackage{fancyhdr}  % Optional, for setting a persistent header/footer
\usepackage{enumitem} 
\usepackage{fancyhdr}

\geometry{a4paper, margin=0.8in}

% --- Custom Footer Setup ---
\pagestyle{fancy}
\fancyhf{} % Clear headers/footers
\fancyfoot[L]{\thepage\ / Volume 1, 2025} 
\fancyfoot[R]{FIBO FRA333 – Kinematics Final Project}
\renewcommand{\headrulewidth}{0pt}
\renewcommand{\footrulewidth}{0.5pt}
% ---------------------------

\begin{document}

\noindent
\begin{tabular}{p{0.90\textwidth} c} 
    \parbox[t]{0.90\textwidth}{ 
        \raggedright 
        Institute of Field Robotics, King Mongkut's University of Technology Thonburi \\
        126 Pracha Uthit Rd, Bang Mot, Thung Khru, Bangkok, Thailand 10140 \\
        Senior Thesis FRAB (FIBO Robotics and Automation: Bachelor) \\
        Copyright $\copyright$ 2021 by FIBO
        \vspace{0.5cm}
    }
    & 
    \raisebox{-1.4cm}
    {
        \includegraphics[width=0.7in, keepaspectratio]{img/fibo_logo.png}
    }
\end{tabular}

\noindent
\rule{\textwidth}{1pt}

\vspace{0.4cm}
\noindent
\Large {\textbf{The study of UR5 to perform the drawing task using torque control}}

\noindent
\rule{\textwidth}{0.3pt}

\begin{minipage}[t]{0.40\textwidth}
    \raggedright
    \normalsize
    Jirattanun Leeudomwong, 66340500009 \\
    Bharut Auasudkit, 66340500040 \\
    Nuttapruet Puttiwarodom, 66340500018 \\
\end{minipage}
\hfill
\vline
\begin{minipage}[t]{0.60\textwidth}
    \raggedright
    \normalsize
    \begin{quote}
        \textbf{Abstract} \\
        Industrial robot manipulator have countless application nowadays. One of the basic task to test the accuracy of the robot manipulator is to perform drawing task and can be done with only position control alone. This project focusing on using UR5 industrial robot manipulator to perform the drawing task. The challenge this project represent is to control the force between pen (end-effector) and canvas using torque control while following a trajectory generated by the image processing method call edge detection. This project running on Gazebo Ignition physic simulation and using ROS2 as a middleware.
        \vspace{0.5cm}

        \textbf{Keywords: Torque control, Dynamics model, Trajectory generation, Image processing, Edge detection, ROS2, Physic simulation}
    \end{quote}
\end{minipage}

\vspace{0.4cm}
\noindent
\rule{\textwidth}{1pt}

\linespread{1.5} 
% ------------------- context zone ------------------- 
\vspace{0.2cm}
\noindent
\large
\textbf{1. Introduction} \\
\textbf{1.1 Objective}  \\
\normalsize
\vspace*{-0.9\baselineskip}
\begin{enumerate}[nosep, itemsep=-2pt] 
    \item To develop the UR5 industrial robot manipulator to perform the drawing task on Gazebo Ignition physic simulation. 
    \item To develop the torque control system that can follow the position, velocity and torque trajectory.
    \item To develop the trajectory generation process that can generated the trajectory from any image.
\end{enumerate}
  
\vspace{8pt}
\noindent
\large 
\textbf{1.2 Scope of the study} \\ 
\normalsize
\vspace*{-0.9\baselineskip} 
\begin{enumerate}[nosep, itemsep=-2pt] 
    \item This project study and test only on the UR5 Gazebo Ignition physic simulation with ROS2 middleware and does not cover the real world application.
    \item This project only coverage drawing task with 1 fix drawing plain and not coverage pen changing, drawing at multiple plain or others additional task.
\end{enumerate}

\large
\noindent
\textbf{2. Literature review} \\
\indent
lorem ipsum

\large
\noindent
\textbf{3. Related knowledge with FRA333} \\
\normalsize
\vspace*{-0.9\baselineskip} 
\begin{enumerate}[nosep, itemsep=-2pt]
    \item Forward Kinematics 
    \item Inverse Kinematics
    \item Differential Kinematics
    \item Dynamics
    \item Trajectory generation
\end{enumerate}

\newpage

\large
\noindent
\textbf{4. System diagram} \\
\normalsize

\large
\noindent
\textbf{5. Study goal} \\
\normalsize
\vspace*{-0.9\baselineskip} 
\begin{enumerate}[nosep, itemsep=-2pt]
    \item Complete UR5 and environment simulation in Gazebo Ignition.
    \item UR5 can be performed drawing task from trajectory.
    \item UR5 can be control and maintain pressure that act to canvas. 
    \item Trajectory can be generated by using any image.
\end{enumerate}
    
\large
\noindent
\textbf{6. Gantt chart} \\
\normalsize

\large
\noindent
\textbf{7. Reference} \\
\normalsize 

\end{document}